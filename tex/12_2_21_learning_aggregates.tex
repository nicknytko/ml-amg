\documentclass{article}

\usepackage{lipsum}
\usepackage{amsfonts}
\usepackage{amsmath}
\usepackage{amsthm}
\usepackage{graphicx}
\graphicspath{{figures/12_2_21_learning_aggregates/}}
\usepackage{epstopdf}
\ifpdf%
  \DeclareGraphicsExtensions{.eps,.pdf,.png,.jpg}
\else
  \DeclareGraphicsExtensions{.eps}
\fi
\usepackage{amsopn}
\DeclareMathOperator{\diag}{diag}
\usepackage{booktabs}
\usepackage{bbm}
\usepackage{bm}
\usepackage{caption}
\usepackage{subcaption}
\usepackage[utf8]{inputenc}
\usepackage[T1]{fontenc}
\usepackage[margin=1in]{geometry}
\usepackage{hyperref}
\usepackage{algorithm}
\usepackage{algpseudocode}

\newcommand{\norm}[1]{\left\lVert#1\right\rVert}
\newcommand{\normtwo}[1]{\left\lVert#1\right\rVert_2}
\newcommand{\abs}[1]{\left\lvert#1\right\rvert}
\newcommand{\mat}[1]{\bm{{#1}}}
\renewcommand{\vec}[1]{\bm{{#1}}}
\newcommand{\lequiv}{\Leftrightarrow}
\newcommand{\bigO}[1]{\mathcal{O}\!\left(#1\right)}
\newcommand{\ceil}[1]{\left\lceil #1 \right\rceil}
\newcommand{\floor}[1]{\left\lfloor #1 \right\rfloor}
\newcommand{\sfrac}[2]{#1/#2}
\newcommand{\hquad}{\enskip}
\newcommand{\expected}[1]{\mathbb{E}\left[#1\right]}
\newcommand{\mspan}[1]{\text{span}\left( #1 \right)}
\newcommand{\prob}[1]{P\left(#1\right)}
\newcommand{\probt}[1]{P\left( \text{#1} \right)}
\newcommand{\condprob}[2]{P\left(#1 \:|\: #2\right)}
\newcommand{\condprobt}[2]{P\left(\text{#1} \:|\: \text{#2}\right)}
\newcommand{\bayes}[2]{\frac{\condprob{#2}{#1}\prob{#1}}{\prob{#2}}}
\newcommand{\bayesx}[3]{\frac{\condprob{#2}{#1}\prob{#1}}{\condprob{#2}{#1}\prob{#1} + \condprob{#2}{#3}\prob{#3}}}
\newcommand{\sech}{\text{sech}}
\newcommand*{\vertbar}{\rule[-1ex]{0.5pt}{2.5ex}}
\newcommand*{\horzbar}{\rule[.5ex]{2.5ex}{0.5pt}}
\newcommand{\vect}[2]{\underline{{#1}}_{{#2}}}
\newcommand{\basisp}[1]{\underline{{p}}_{{#1}}}
\newcommand{\basisq}[1]{\underline{{q}}_{{#1}}}
\newcommand{\coeff}[1]{\underline{{a}}_{{#1}}}
\newcommand{\bestfit}{\underline{\bar{x}}}
\newcommand{\grad}{\nabla}
\newcommand{\laplace}{\Delta}
\newcommand{\setbar}{\:\middle|\:}
\renewcommand{\div}{\grad \cdot}
\renewcommand{\Re}{\text{Re}}

\begin{document}

\begin{figure}[b]
  \centering
  \begin{subfigure}[b]{\textwidth}
    \centering
    \includegraphics[height=4in]{isotropic_lloyd.pdf}
    \caption{Lloyd Aggregates}
  \end{subfigure}
  \hfill
  \begin{subfigure}[b]{0.49\textwidth}
    \centering
    \includegraphics[width=\textwidth]{isotropic_full.pdf}
    \caption{ML aggregates and interpolation, both networks trained concurrently}
  \end{subfigure}
  \begin{subfigure}[b]{0.49\textwidth}
    \centering
    \includegraphics[width=\textwidth]{isotropic_full_separate.pdf}
    \caption{ML aggregates and interpolation, networks trained separately}
  \end{subfigure}
  \caption{Aggregates for isotropic 2D Poisson, $16 \times 16$ grid.}
  \label{fig:2d_isotropic}
\end{figure}

\begin{figure}[b]
  \centering
  \begin{subfigure}[b]{\textwidth}
    \centering
    \includegraphics[height=4in]{anisotropic_lloyd.pdf}
    \caption{Lloyd Aggregates}
  \end{subfigure}
  \hfill
  \begin{subfigure}[b]{0.49\textwidth}
    \centering
    \includegraphics[width=\textwidth]{anisotropic_full.pdf}
    \caption{ML aggregates and interpolation, both networks trained concurrently}
  \end{subfigure}
  \begin{subfigure}[b]{0.49\textwidth}
    \centering
    \includegraphics[width=\textwidth]{anisotropic_full_separate.pdf}
    \caption{ML aggregates and interpolation, networks trained separately}
  \end{subfigure}
  \caption{Aggregates for isotropic 2D Poisson, $16 \times 16$ grid.}
  \label{fig:2d_anisotropic}
\end{figure}

\begin{figure}[b]
  \centering
  \begin{subfigure}[t]{0.49\textwidth}
    \centering
    \includegraphics[width=\textwidth]{isotropic_large_lloyd.pdf}
    \caption{Lloyd Aggregates}
  \end{subfigure}
  \begin{subfigure}[t]{0.49\textwidth}
    \centering
    \includegraphics[width=\textwidth]{isotropic_large_full.pdf}
    \caption{ML aggregates and interpolation, both networks trained concurrently}
  \end{subfigure}
  \caption{Aggregates for isotropic 2D Poisson, $20\times 20$ grid.}
  \label{fig:2d_anisotropic}
\end{figure}

\end{document}
